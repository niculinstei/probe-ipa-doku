\chapter{Kontrollieren}\label{ch:kontrollieren}
Dieses Kapitel bietet Rahmen für die Arbeiten, welche in der IPERKA-Phase <<Kontrollieren>> angefallen sind. In dieser Phase wird die Qualität der Umsetzung geprüft. Es werden alle im Test und Qualitätssicherungskonzept definierten Checks durchgeführt. Falls Fehler auftreten, werden diese behoben.
\section{Durchführen der Tests}
In diesem Abschnitt sind die Ergebnisse der Tests dokumentiert. Die durchgeführten Tests sind im Testkonzept in Kapitel \ref{sec:testkonzept} aufgeführt.
\begin{longtable}{|p{.50\textwidth}|p{.50\textwidth}|}
	\hline
	\textbf{Testdatum} & 18.11.2024 \& 20.11.2024\\
	\hline
	\textbf{Testperson} & Niculin Steiner\\
	\hline
\end{longtable}

\subsection{Unit-Tests}
Alle geschriebenen Unit-Tests sollen erfolgreich durchlaufen.
Folgende 2 Services hatten Logik welche mit Unit-Tests abgedeckt werden musste:
\begin{itemize}
	\item FuturaeAdminApiEnrollmentServiceImpl.java\\
	*Die rot umrahmten Tests wurden im Rahmen der Probe-IPA erstellt.
		\begin{figure}[H]
			\begin{center}
				\includegraphics[width=0.8\textwidth]{ressourcen/unittestapi}
				\caption[Unit-Test Resultate FuturaeAdminApiEnrollmentServiceImpl.java]{Unit-Test Resultate FuturaeAdminApiEnrollmentServiceImpl.java}\label{fig:test-admin}
			\end{center}
		\end{figure}
	\item Airlock2FAAdminService.java
		\begin{figure}[H]
			\begin{center}
				\includegraphics[width=1.0\textwidth]{ressourcen/testadmin}
				\caption[Unit-Test Resultate Airlock2FAAdminService.java]{Unit-Test Resultate Airlock2FAAdminService.java}\label{fig:unittest-api}
			\end{center}
		\end{figure}
\end{itemize}	
\subsection{REST-Integration-Tests}
Alle geschriebenen REST-Integration-Tests sollen erfolgreich durchlaufen.
\begin{figure}[H]
	\begin{center}
		\includegraphics[width=1.0\textwidth]{ressourcen/resttest}
		\caption[REST-Integration-Tests Resultate]{REST-Integration-Tests Resultate}\label{fig:rest-tests}
	\end{center}
\end{figure}
\subsection{UI-Integration-Tests}
Alle geschriebenen UI-Integration-Tests sollen erfolgreich durchlaufen.
\begin{figure}[H]
	\begin{center}
		\includegraphics[width=1.1\textwidth]{ressourcen/uitests}
		\caption[UI-Integration-Tests Resultate]{UI-Integration-Tests Resultate}\label{fig:ui-tests}
	\end{center}
\end{figure}
\subsection{Manuelle Tests}
Sowie definiert in Abschnitt \ref{subsec:mtests} werden die manuellen Tests durchgeführt. Der Teststatus wird mit folgenden Symbolen dargestellt:\\

\textcolor{green}{\checkmark}: Test erfolgreich \\

\textcolor{red}{\ding{55}}   : Test fehlgeschlagen\\
\\
In der folgenden Tabelle sind die Resultate der Tests aufgelistet: \newpage
\begin{longtable}{|p{.10\textwidth}|p{.40\textwidth}|p{.40\textwidth}|p{.10\textwidth}|}
	\hline
	\textbf{Testfall} & \textbf{Erwartetes Resultat} & \textbf{Tatsächliches Resultat} &\textbf{Status} \\ \hline
	M1 & Es öffnet sich ein Popup (kein Browserpupop), mit dem Aktivierungscode.  & Es öffnet sich ein Popup (kein Browserpupop), mit dem Aktivierungscode. &  \textcolor{green}{\checkmark} \\ \hline 
	M2 & Der Button wird nicht angezeigt. Der Admin darf auch via einen alternativen REST-Client, wie z.B Postman, den Aktivierungscode nicht bekommen. & Der Button wird nicht angezeigt. Der Admin hat auch via einen alternativen REST-Client, wie z.B Postman, den Aktivierungscode nicht bekommen. & \textcolor{green}{\checkmark} \\ \hline 
	M3 & Der Button wird nicht angezeigt.  &  Der Button wird nicht angezeigt.   & \textcolor{green}{\checkmark} \\ \hline 
\end{longtable}
\noindent Alle Tests konnten erfolgreich durchgeführt werden. Somit funktioniert das neue Feature wie erwartet, und es mussten keine Fehler behoben werden.

\section{Qualitätssicherung}
Die Qualität wurde zum Teil manuell, sowie automatisiert durchgeführt.
\subsection{Manuelle Refactorings}
Um Schreibfehler und andere unschönheiten im Code zu finden und zu verbessern wurde nach dem Implementieren, der ganze, neu geschriebene Code nochmals durchgegangen und kleine Findings wurden direkt behoben oder refactored.

\subsection{Automatisierte Code-Analyse}
Die Qualitätssicherung wurde auch während der Entwicklung fortlaufend sichergestellt. Jeder Commit, welcher auf Gerrit landet löst automatisch die Build Pipeline auf dem Jenkins aus. Diese führt neben den ganzen Code-Tests auch statische Code-Analysen durch. Diese sind:
\begin{itemize}
	\item SpotBugs
	\item PMD
	\item CheckStyle
\end{itemize}
\newpage
Resultate des letzten, finalen Commits:
\begin{figure}[H]
	\begin{center}
		\includegraphics[width=0.8 \textwidth]{ressourcen/codeana}
		\caption[Resultate Code-Analyse]{Resultate der Code-Analyse Tasks auf Jenkins}\label{fig:codeana}
	\end{center}
\end{figure}
\noindent Wie im obigen Bild zusehen, sind alle Checks für diesen Build erfolgreich durchgelaufen. Dies bedeutet die Qualitätssicherung ist so gut wie möglich sichergestellt.
