\chapter{Auswerten}\label{ch:auswerten}
Dieses Kapitel bietet Platz für die letzte IPERKA-Phase namens <<Auswerten>>. Die folgenden Abschnitte enthalten eine persönliche Reflexion, aufgeteilt in die verschiedenen Bereiche der Arbeit. Das Schlusswort rundet das Ganze ab.
\section{Projektmanagement}
Mit der Projektmanagementmethode IPERKA finde ich, habe ich eine gute Entscheidung getroffen. Es strukturiert nicht nur das eigentliche Vorgehen während der Arbeit, sondern auf der Dokumentation wurde damit eine gute Struktur aufgesetzt. In dem ich die Dokumentation nach IPERKA gliederte, hatte ich von Beginn an eine grobe Skizze, wie die Dokumentation aussehen soll. Die Zeitplanung wurde erst in der Planungs-Phase erstellt, was etwas unschön an IPERKA ist, da zuvor schon diverse Arbeiten in der ersten Phase getätigt werden. Allerdings ist es auch wichtig und zwingend sich vor dem Planen über das Projekt zu informieren. 

\section{Zeitmanagement}
Die Zeitplanung ist mir über die gesamte Probe-IPA sehr gut gelungen und ich bin wie geplant fertig geworden. Durch das Einteilen der gesamten Arbeit in kleine, kompakte Arbeitspakete konnte ein übersichtlicher Zeitplan erstellt werden. Beim Zeitplan habe ich mich für das GANTT-Diagramm entschieden. Dies hat sich sehr bewährt. Da ich ihn auch nach den IPERKA-Phasen gegliedert habe, gab es auch hier schon eine gewisse Grundstruktur. Der Zeitplan hatte 2 Stunden Blöcke. Da ich auch Arbeiten hatte, welche nur 1h gedauert haben, führte ich die Option ein, dass ein 2 Stunden Block mit einem <</>> halbiert werden kann. Dies war zu Beginn etwas unübersichtlich. Es hat sich allerdings mit dem fortlaufenden ausfüllen des Zeitplans verbessert. Das GANTT-Diagramm gab mir zudem immer einen sehr guten Überblick wo ich stehe, welche arbeiten noch zu tun sind und was ich schon erledigt habe. Ich würde die Zeitplanung grössten Teils wieder so machen. Speziell für das Testkonzept brauchte ich allerdings nicht soviel Zeit wie geplant. Dafür würde ich für das Testschreiben sicher genügend Zeit einplanen. 
\newpage
\section{Arbeitsprozess}
Nebst dem Programmieren ist auch die Dokumentation in Form dieses Berichts ein wesentlicher Teil der Probe-IPA. Damit alle Arbeiten möglichst genau und effizient dokumentiert werden konnten, habe ich parallel zu der Entwicklung dokumentiert. Dies hatte vor allem den Vorteil das die Erinnerung an das geleistete noch sehr präsent ist. \\
Durch das regelmässige Nachführen des Zeitplans während des ganzen Prozesses, hatte ich immer den vollen Überblick. Die investierten 15h in die Planung haben sich sehr gelohnt, dadurch war ich bei der Umsetzung wesentlich schneller und hatte mehr Zeit um die Dokumentation fortlaufend zu ergänzen. Das diese Dokumentation mit Latex geschrieben wurde hat sich sehr bewährt. So konnte ich mich voll und ganz auf das schreiben fokussieren und Latex hat mir das ganze Formatieren übernommen. Dank dem Template, welches bereits von der Ergon zur Verfügung stand, hatte ich bereits eine sehr gute Grundlage. Natürlich musste ich das Template noch um diverse Punkte erweitern und auf meine Bedürfnisse anpassen. Was allerdings in Latex nicht so gut klappt wie bspw. im Word ist die Rechtschreib- und vor allem die Grammatikprüfung. Für die Rechtschreibprüfung konnte ich eine Erweiterung herunterladen welche mir die Wörter auf Schreibfehler prüft. Für die Grammatikprüfung konnte ich aber keine Lösung finden.
\section{Ergebnis der Probe-IPA}
Die Anforderungen an das Resultat der Probe-IPA konnte alle umgesetzt werden. In der Adminapp sich kann der Adminnutzer den 16-stelligen Aktivierungscode per Knopfdruck anzeigen lassen. Hat er keine Berechtigung wird ihm diese Option nicht angeboten. Dabei wurden in der SPA nur bereits bestehende IAM-Komponenten verwendet, was zur Folge hat, dass sich das neue Feature im UI und UX Bereich gleich verhält wie das bereits bestehende.\\
Es wurden alle neuen Funktionalitäten mit Unit-Tests, sowie REST- und UI-Integration-Tests abgedeckt.
Des weiteren wurde die Kundendokumentation(Abschnitt in diesem Bericht) um das neue Feature erweitert.

\section{Schlusswort}
Die 10 Tage während der Probe-IPA waren sehr intensiv aber auch interessant und hilfreich im Ausblick auf die richtige IPA. Durch den Zeitplan, welcher eingeplante Puffer hatte, kam ich nie in einen extremen Stress. Dies lies mich etwas ruhiger und konzentrierter arbeiten. Auch wenn die Probe-IPA durch das viele Dokumentieren, genauen Planen und der Arbeit alleine stark von der Realität im Berufsalltag abweicht, konnte ich mich sehr schnell daran gewöhnen. Durch den Einsatz von IPERKA konnte ich überraschend gut strukturiert arbeiten. Mit meiner Umgesetzten Lösung bin ich zufrieden. Ich konnte alle Anforderungen umsetzen, was mich schon mal positiv stimmt. Für die richtige IPA werden ich versuchen noch etwas genauer zu Planen und mir dort auch die bestehende Testinfrastruktur im Code anzuschauen. So kann ich auch dort unvorhergesehene Aufwände minimieren. Ich hatte zum Teil kleinere Probleme mit meinem Latex Editor (TeXstudio) / wusste Teilweise nicht wie was geht, und verlor Zeit beim recherchieren. Diese Zeit würde ich beim nächsten mal schon vor der Probe-IPA investieren.\\
Insgesamt bin ich aber sehr zufrieden sowohl mit dem Gesamtprodukt als auch mit meinem Prozess dahin.


