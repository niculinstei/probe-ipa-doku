\chapter{Arbeitsprotokoll}\label{ch:arbeitsprotokoll}
\renewcommand{\arraystretch}{1.5}
\begin{longtable}{p{.22\textwidth}|p{.78\textwidth}}
    \hline
    \textbf{Datum}                       & 06.11.2024\\
    \hline
    \textbf{Bearbeitete Arbeitspakete}   & 1.1, 1.2, 2.1, 2.2, 7.1, 7.2, 7.3\\
    \hline
    \textbf{Arbeitszeit}                 & 8h \\
    \hline
    \textbf{Überzeit}                    & 0 \\
    \hline
    \textbf{Vergleich mit dem Zeitplan}  & Da ich den Zeitplan noch nicht fertig erstellt habe, kann ich für heute keinen Vergleich ziehen. \\
    \hline
    \textbf{Erfolge und Probleme}        & Zu Beginn wusste ich nicht genau wie ich am besten vorgehe resp. was ich zuerst angehe, da hat mir das vorhandene Template einen sehr guten Leitfaden gegeben. Und so habe ich begonnen alles der Reihe nach auszufüllen/ zu dokumentieren. Und bin am Schluss weiter gekommen als gedacht.\\
    \hline
    \textbf{Tagesreflexion}              & Heute bin ich sehr gut voran gekommen. Ich konnte bereits den Teil 1 der Dokumentation abschliessen und mit den Arbeitspaketen beginnen.
    \\
    \hline
    \textbf{In Anspruch genommene Hilfe} & Fragen an Pascal bezüglich der Aufgabenstellung. War mir unsicher, wo genau die Kundendoku hin muss. Jetzt weiss ich, dass es reicht, wenn ich sie im Anhang anhänge.\\
    \hline
\end{longtable}\label{tab:arbeitsprotokoll-tag1} 

\newpage

\begin{longtable}{p{.22\textwidth}|p{.78\textwidth}}
	\hline
	\textbf{Datum}                       & 07.11.2024 \\
	\hline
	\textbf{Bearbeitete Arbeitspakete}   & 2.1, 2.2, 2.3, 2.4 \\
	\hline
	\textbf{Arbeitszeit}                 & 8h \\
	\hline
	\textbf{Überzeit}                    & 0 \\
	\hline
	\textbf{Vergleich mit dem Zeitplan}  & Eine Stunde voraus \\
	\hline
	\textbf{Erfolge und Probleme}        & Beim Zeitplan hatte das Template nicht richtig funktioniert. Da kam ein bisschen extra Aufwand dazu, da ich aber ansonsten etwas schenller war hat sich das wieder Kompensiert. Ich konnte bereits heute mit dem SPA Lösungskonzept beginnen. Da gestern der Zeitplan noch nicht stand, erwähne ich es heute: Ein weiterer Erfolg, ich konnte
Meilenstein A (Informieren)	gestern erfolgreich und überpünktlich abschliessen.\\
	\hline
	\textbf{Tagesreflexion}              & Ich bin auch heute wieder sehr gut voran gekommen, und bin somit dem Zeitplan eine Stunde voraus. Dies finde ich sehr angenehm, denn es lässt einem etwas ruhiger und weniger gestresst Arbeiten.\\
	\hline
	\textbf{In Anspruch genommene Hilfe} & keine\\
	\hline
\end{longtable}\label{tab:arbeitsprotokoll-tag2}

\newpage

\begin{longtable}{p{.22\textwidth}|p{.78\textwidth}}
	\hline
	\textbf{Datum}                       & 08.11.2024 \\
	\hline
	\textbf{Bearbeitete Arbeitspakete}   & 2.4, 2.5 \\
	\hline
	\textbf{Arbeitszeit}                 & 8h \\
	\hline
	\textbf{Überzeit}                    & 0 \\
	\hline
	\textbf{Vergleich mit dem Zeitplan}  & Ich hatte für das Lösungskonzept der SPA eine Stunde länger als gedacht. Dadurch bin ich gerade genau im Zeitplan, da ich für das Backend weniger Zeit als geplant brauchte. \\
	\hline
	\textbf{Erfolge und Probleme}        & Heute hatte ich eine kurze Zeit Probleme mit Latex. Dies hat mich etwas Zeit gekostet. Ansonsten bin ich gut voran gekommen und auf Kurs.
	\\
	\hline
	\textbf{Tagesreflexion}              & Heute habe ich am Lösungskonzept der SPA und dem Testkonzept gearbeitet. Das für die SPA konnte ich bereits abschliessen. Am Montag geht es dann weiter mit dem Testkonzept, mit welchem ich heute schon begonnen habe. Zudem hatte ich am Morgen meinen ersten Expertenbesuch, welcher im Rahmen der Probe-IPA von Bernd durchgeführt worden ist.
	\\
	\hline
	\textbf{In Anspruch genommene Hilfe} & keine \\
	\hline
\end{longtable}\label{tab:arbeitsprotokoll-tag3}
\newpage

\begin{longtable}{p{.22\textwidth}|p{.78\textwidth}}
	\hline
	\textbf{Datum}                       & 11.11.2024 \\
	\hline
	\textbf{Bearbeitete Arbeitspakete}   & 3.1, 4.1 \\
	\hline
	\textbf{Arbeitszeit}                 & 8h \\
	\hline
	\textbf{Überzeit}                    & 0 \\
	\hline
	\textbf{Vergleich mit dem Zeitplan}  & Ich konnte etwas früher mit der Backend Implementation beginnen.  \\
	\hline
	\textbf{Erfolge und Probleme}        & Ich hatte heute etwas Schwierigkeiten Code stellen und Regexe in Latex einzufügen. Dies hat mich etwas Zeit gekostet, ich konnte es aber lösen.
	\\
	\hline
	\textbf{Tagesreflexion}              & Heute habe ich den Rest Endpunkt implementiert und dokumentiert. Ich konnte mit der weiteren Logik bereits beginnen. 
	\\
	\hline
	\textbf{In Anspruch genommene Hilfe} & keine \\
	\hline
\end{longtable}\label{tab:arbeitsprotokoll-tag4}
\newpage

\begin{longtable}{p{.22\textwidth}|p{.78\textwidth}}
	\hline
	\textbf{Datum}                       & 13.11.2024 \\
	\hline
	\textbf{Bearbeitete Arbeitspakete}   & 4.1, 4.2 \\
	\hline
	\textbf{Arbeitszeit}                 & 8h \\
	\hline
	\textbf{Überzeit}                    & 0 \\
	\hline
	\textbf{Vergleich mit dem Zeitplan}  & Ich bin dem Zeitplan etwas voraus. Ich konnte heute bereits mit den Tests für das Backend beginnen.\\
	\hline
	\textbf{Erfolge und Probleme}        & Ich bin heute sehr gut voran gekommen, und konnte das Backend fertig implementieren. So konnte ich bereits mit den Tests beginnen. Allerdings hatte ich da zu Beginn noch kleine Schwirigkeiten mit Wiremock. Diese liessen sich aber mit etwas Gedult beheben.
	\\
	\hline
	\textbf{Tagesreflexion}              & Heute habe ich die Hauptfunktionalitäten des Backends implementiert. Und bereits mit den Tests begonnen. Es war ein sehr produktiver Tag.
	\\
	\hline
	\textbf{In Anspruch genommene Hilfe} & keine \\
	\hline
\end{longtable}\label{tab:arbeitsprotokoll-tag5}
\newpage

\begin{longtable}{p{.22\textwidth}|p{.78\textwidth}}
	\hline
	\textbf{Datum}                       & 14.11.2024 \\
	\hline
	\textbf{Bearbeitete Arbeitspakete}   & 4.1, 4.2, 4.3 \\
	\hline
	\textbf{Arbeitszeit}                 & 8h \\
	\hline
	\textbf{Überzeit}                    & 0 \\
	\hline
	\textbf{Vergleich mit dem Zeitplan}  & Ich bin im Zeitplan bisschen voraus, und konnte bereits mit der SPA-Implementierung beginnen.\\
	\hline
	\textbf{Erfolge und Probleme}        & Heute bin ich sehr weit gekommen ich konnte das Backen abschliessen inkl. Tests. Im Frontend konnt ich bereits den Durchstich erzielen.
	\\
	\hline
	\textbf{Tagesreflexion}              & Es war ein sehr produktiver Tag. Am morgen war ich beschäftigt mit der Dokumentation des Backends und den Tests für das Backend. Am späteren Nachmittag begann ich dann mit der SPA.
	\\
	\hline
	\textbf{In Anspruch genommene Hilfe} & Frage an Pascal bzgl. Tiefe der Dokumentation. \\
	\hline
\end{longtable}\label{tab:arbeitsprotokoll-tag6}
\newpage

\begin{longtable}{p{.22\textwidth}|p{.78\textwidth}}
	\hline
	\textbf{Datum}                       & 15.11.2024 \\
	\hline
	\textbf{Bearbeitete Arbeitspakete}   & ... \\
	\hline
	\textbf{Arbeitszeit}                 & ... \\
	\hline
	\textbf{Überzeit}                    & ... \\
	\hline
	\textbf{Vergleich mit dem Zeitplan}  & ... \\
	\hline
	\textbf{Erfolge und Probleme}        & ...
	\\
	\hline
	\textbf{Tagesreflexion}              & ...
	\\
	\hline
	\textbf{In Anspruch genommene Hilfe} & ... \\
	\hline
\end{longtable}\label{tab:arbeitsprotokoll-tag7}
\newpage


