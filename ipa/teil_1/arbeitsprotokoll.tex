\chapter{Arbeitsprotokoll}\label{ch:arbeitsprotokoll}
\renewcommand{\arraystretch}{1.5}
\begin{longtable}{p{.22\textwidth}|p{.78\textwidth}}
    \hline
    \textbf{Datum}                       & 06.11.2024\\
    \hline
    \textbf{Bearbeitete Arbeitspakete}   & 1.1, 1.2, 2.1, 2.2, 7.1, 7.2, 7.3\\
    \hline
    \textbf{Arbeitszeit}                 & 8h \\
    \hline
    \textbf{Überzeit}                    & 0 \\
    \hline
    \textbf{Vergleich mit dem Zeitplan}  & Da ich den Zeitplan noch nicht fertig erstellt habe, kann ich für heute keinen Vergleich ziehen. \\
    \hline
    \textbf{Erfolge und Probleme}        & Zu Beginn wusste ich nicht genau wie ich am besten vorgehe resp. was ich zuerst angehe, da hat mir das vorhandene Template einen sehr guten Leitfaden gegeben. Und so habe ich begonnen alles der Reihe nach auszufüllen/ zu dokumentieren. Und bin am Schluss weiter gekommen als gedacht.\\
    \hline
    \textbf{Tagesreflexion}              & Heute bin ich sehr gut voran gekommen. Ich konnte bereits den Teil 1 der Dokumentation abschliessen und mit den Arbeitspaketen beginnen.
    \\
    \hline
    \textbf{In Anspruch genommene Hilfe} & Fragen an Pascal bezüglich der Aufgabenstellung. War mir unsicher, wo genau die Kundendokumentation hin muss. Jetzt weiss ich, dass es reicht, wenn ich sie im Anhang anhänge.\\
    \hline
\end{longtable}\label{tab:arbeitsprotokoll-tag1} 

\newpage

\begin{longtable}{p{.22\textwidth}|p{.78\textwidth}}
	\hline
	\textbf{Datum}                       & 07.11.2024 \\
	\hline
	\textbf{Bearbeitete Arbeitspakete}   & 2.1, 2.2, 2.3, 2.4 \\
	\hline
	\textbf{Arbeitszeit}                 & 8h \\
	\hline
	\textbf{Überzeit}                    & 0 \\
	\hline
	\textbf{Vergleich mit dem Zeitplan}  & Eine Stunde voraus \\
	\hline
	\textbf{Erfolge und Probleme}        & Beim Zeitplan hatte das Template nicht richtig funktioniert. Da kam ein bisschen extra Aufwand dazu, da ich aber ansonsten etwas schneller war hat sich das wieder Kompensiert. Ich konnte bereits heute mit dem SPA Lösungskonzept beginnen. Da gestern der Zeitplan noch nicht stand, erwähne ich es heute: Ein weiterer Erfolg, ich konnte
Meilenstein A (Informieren)	gestern erfolgreich und überpünktlich abschliessen.\\
	\hline
	\textbf{Tagesreflexion}              & Ich bin auch heute wieder sehr gut voran gekommen, und bin somit dem Zeitplan eine Stunde voraus. Dies finde ich sehr angenehm, denn es lässt einem etwas ruhiger und weniger gestresst Arbeiten.\\
	\hline
	\textbf{In Anspruch genommene Hilfe} & keine\\
	\hline
\end{longtable}\label{tab:arbeitsprotokoll-tag2}

\newpage

\begin{longtable}{p{.22\textwidth}|p{.78\textwidth}}
	\hline
	\textbf{Datum}                       & 08.11.2024 \\
	\hline
	\textbf{Bearbeitete Arbeitspakete}   & 2.4, 2.5, 7.4 \\
	\hline
	\textbf{Arbeitszeit}                 & 8h \\
	\hline
	\textbf{Überzeit}                    & 0 \\
	\hline
	\textbf{Vergleich mit dem Zeitplan}  & Ich hatte für das Lösungskonzept der SPA eine Stunde länger als gedacht. Dadurch bin ich gerade genau im Zeitplan, da ich für das Backend weniger Zeit als geplant brauchte. \\
	\hline
	\textbf{Erfolge und Probleme}        & Heute hatte ich eine kurze Zeit Probleme mit Latex. Dies hat mich etwas Zeit gekostet. Ansonsten bin ich gut voran gekommen und auf Kurs.
	\\
	\hline
	\textbf{Tagesreflexion}              & Heute habe ich am Lösungskonzept der SPA und dem Testkonzept gearbeitet. Das für die SPA konnte ich bereits abschliessen. Am Montag geht es dann weiter mit dem Testkonzept, mit welchem ich heute schon begonnen habe. Zudem hatte ich am Morgen meinen ersten Expertenbesuch, welcher im Rahmen der Probe-IPA von Bernd durchgeführt worden ist.
	\\
	\hline
	\textbf{In Anspruch genommene Hilfe} & keine \\
	\hline
\end{longtable}\label{tab:arbeitsprotokoll-tag3}
\newpage

\begin{longtable}{p{.22\textwidth}|p{.78\textwidth}}
	\hline
	\textbf{Datum}                       & 11.11.2024 \\
	\hline
	\textbf{Bearbeitete Arbeitspakete}   & 3.1, 4.1 \\
	\hline
	\textbf{Arbeitszeit}                 & 8h \\
	\hline
	\textbf{Überzeit}                    & 0 \\
	\hline
	\textbf{Vergleich mit dem Zeitplan}  & Ich konnte etwas früher mit der Backend Implementation beginnen.  \\
	\hline
	\textbf{Erfolge und Probleme}        & Ich hatte heute etwas Schwierigkeiten Code stellen und Regexe in Latex einzufügen. Dies hat mich etwas Zeit gekostet, ich konnte es aber lösen.
	\\
	\hline
	\textbf{Tagesreflexion}              & Heute habe ich den Rest Endpunkt implementiert und dokumentiert. Ich konnte mit der weiteren Logik bereits beginnen. 
	\\
	\hline
	\textbf{In Anspruch genommene Hilfe} & keine \\
	\hline
\end{longtable}\label{tab:arbeitsprotokoll-tag4}
\newpage

\begin{longtable}{p{.22\textwidth}|p{.78\textwidth}}
	\hline
	\textbf{Datum}                       & 13.11.2024 \\
	\hline
	\textbf{Bearbeitete Arbeitspakete}   & 4.1, 4.2 \\
	\hline
	\textbf{Arbeitszeit}                 & 8h \\
	\hline
	\textbf{Überzeit}                    & 0 \\
	\hline
	\textbf{Vergleich mit dem Zeitplan}  & Ich bin dem Zeitplan etwas voraus. Ich konnte heute bereits mit den Tests für das Backend beginnen.\\
	\hline
	\textbf{Erfolge und Probleme}        & Ich bin heute sehr gut voran gekommen, und konnte das Backend fertig implementieren. So konnte ich bereits mit den Tests beginnen. Allerdings hatte ich da zu Beginn noch kleine Schwierigkeiten mit Wiremock. Diese liessen sich aber mit etwas Geduld beheben.
	\\
	\hline
	\textbf{Tagesreflexion}              & Heute habe ich die Hauptfunktionalitäten des Backends implementiert. Und bereits mit den Tests begonnen. Es war ein sehr produktiver Tag.
	\\
	\hline
	\textbf{In Anspruch genommene Hilfe} & keine \\
	\hline
\end{longtable}\label{tab:arbeitsprotokoll-tag5}
\newpage

\begin{longtable}{p{.22\textwidth}|p{.78\textwidth}}
	\hline
	\textbf{Datum}                       & 14.11.2024 \\
	\hline
	\textbf{Bearbeitete Arbeitspakete}   & 4.1, 4.2, 4.3 \\
	\hline
	\textbf{Arbeitszeit}                 & 8h \\
	\hline
	\textbf{Überzeit}                    & 0 \\
	\hline
	\textbf{Vergleich mit dem Zeitplan}  & Ich bin im Zeitplan bisschen voraus, und konnte bereits mit der SPA-Implementierung beginnen.\\
	\hline
	\textbf{Erfolge und Probleme}        & Heute bin ich sehr weit gekommen ich konnte das Backend abschliessen inkl. Tests. Im Frontend konnte ich bereits den Durchstich erzielen.
	\\
	\hline
	\textbf{Tagesreflexion}              & Es war ein sehr produktiver Tag. Am morgen war ich beschäftigt mit der Dokumentation des Backends und den Tests für das Backend. Am späteren Nachmittag begann ich dann mit der SPA.
	\\
	\hline
	\textbf{In Anspruch genommene Hilfe} & Frage an Pascal bzgl. Tiefe der Dokumentation. \\
	\hline
\end{longtable}\label{tab:arbeitsprotokoll-tag6}
\newpage

\begin{longtable}{p{.22\textwidth}|p{.78\textwidth}}
	\hline
	\textbf{Datum}                       & 15.11.2024 \\
	\hline
	\textbf{Bearbeitete Arbeitspakete}   & 4.3, 4.4, 7.4 \\
	\hline
	\textbf{Arbeitszeit}                 & 8h \\
	\hline
	\textbf{Überzeit}                    & 0 \\
	\hline
	\textbf{Vergleich mit dem Zeitplan}  & Im Vergleich zum Zeitplan bin ich ein bisschen voraus, und konnte bereits heute mit den Selenium Integration-Tests beginnen.\\
	\hline
	\textbf{Erfolge und Probleme}        & Ich konnte wieder ein Arbeitspaket vor der geplanten Zeit abschliessen. Ich hatte zu Beginn etwas Probleme mit den Selenium-Tests, bzw. ich brauchte einige Zeit, bis ich die richtigen Selektoren für die jeweiligen HTML-Elemente gefunden habe.
	\\
	\hline
	\textbf{Tagesreflexion}              & Heute bin ich gut vorangekommen und konnte die Implementation in der SPA abschliessen, und bereits mit den Selenium-Tests beginnen. In der ersten Hälfte des Nachmittags hatte ich den 2. Expertenbesuch, welcher wieder durch Bernd simuliert wurde. Dieser lief auch gut, sprich ich konnte alles zeigen, was gefordert wurde.
	\\
	\hline
	\textbf{In Anspruch genommene Hilfe} & keine\\
	\hline
\end{longtable}\label{tab:arbeitsprotokoll-tag7}
\newpage

\begin{longtable}{p{.22\textwidth}|p{.78\textwidth}}
	\hline
	\textbf{Datum}                       & 18.11.2024 \\
	\hline
	\textbf{Bearbeitete Arbeitspakete}   & 4.4, 4.5, 5.1\\
	\hline
	\textbf{Arbeitszeit}                 & 8h \\
	\hline
	\textbf{Überzeit}                    & 0 \\
	\hline
	\textbf{Vergleich mit dem Zeitplan}  & Da ich für die Selenium-Tests etwas länger gebraucht habe, bin ich dem Zeitplan nicht mehr voraus. Dies ist allerdings nicht dramatisch, denn ich bin jetzt genau soweit ich geplant habe.\\
	\hline
	\textbf{Erfolge und Probleme}        & Bei den Selenium-Tests hatte ich heute einige Probleme. Ich verschwendete zu viel Zeit, für einen zu komplizierten Lösungsweg, den man auch anders lösen könnte. Schlussendlich bin ich dann auf die einfachere andere Lösung gekommen. Nichts desto trotz konnte ich heute den Meilenstein <<Realisiern>> pünktlich abschliessen. Und mit dem nächsten Schritt dem <<Kontrollieren>> weiter machen.
	\\
	\hline
	\textbf{Tagesreflexion}              &  Heute war ich vom Morgen bis zur Hälfte des Nachmittags mit den Selenium-Tests beschäftigt. Danach startete ich mit der neuen IPERKA-Phase <<Kontrollieren>>. Bei dieser bin ich schon sehr weit gekommen. Ich konnte alle manuellen Tests, Unit- und Integration-Tests bereits ausführen.
	\\
	\hline
	\textbf{In Anspruch genommene Hilfe} & keine \\
	\hline
\end{longtable}\label{tab:arbeitsprotokoll-tag8}
\newpage

\begin{longtable}{p{.22\textwidth}|p{.78\textwidth}}
	\hline
	\textbf{Datum}                       & 20.11.2024 \\
	\hline
	\textbf{Bearbeitete Arbeitspakete}   & 5.1, 5.2, 5.3\\
	\hline
	\textbf{Arbeitszeit}                 & 8h \\
	\hline
	\textbf{Überzeit}                    & 0 \\
	\hline
	\textbf{Vergleich mit dem Zeitplan}  & Ich bin super im Zeitplan und mit den letzten Arbeiten der Dokumentation beschäftigt. \\
	\hline
	\textbf{Erfolge und Probleme}        & Heute konnte ich die Arbeitspakete 5.1 und 5.2 unter der geplanten Zeit abschliessen und mit 5.3 beginnen. 
	\\
	\hline
	\textbf{Tagesreflexion}              & Heute war der Tag, der mir bisher am wenigsten Freude bereitet hat. Ich war ab der zweiten Hälfte des Morgen viel mit Dokumentation / Code durchlesen und Schreibfehler verbessern beschäftigt. Ich allerdings schon sehr weit gekommen, so das morgen nicht mehr diesbezüglich gemacht werden muss.
	\\
	\hline
	\textbf{In Anspruch genommene Hilfe} & Fragen an Pascal, ob und wie ich den geschrieben Code im Anhang angeben darf. Ich darf den Code anhängen, am besten der <<Diff>> zum Master des letzten Patchsets.\\
	\hline
\end{longtable}\label{tab:arbeitsprotokoll-tag9}
\newpage

\begin{longtable}{p{.22\textwidth}|p{.78\textwidth}}
	\hline
	\textbf{Datum}                       & 21.11.2024 \\
	\hline
	\textbf{Bearbeitete Arbeitspakete}   &  5.3, 6.2, 7.5\\
	\hline
	\textbf{Arbeitszeit}                 & 8h \\
	\hline
	\textbf{Überzeit}                    & 0 \\
	\hline
	\textbf{Vergleich mit dem Zeitplan}  & Der Zeitplan ging auf. Ich bin genau rechtzeitig mit der Probe-IPA fertig geworden.
	\\
	\hline
	\textbf{Erfolge und Probleme}        & Heute war der letzte Tag. Der grösste Erfolg war sicher, dass ich die Probe-IPA erfolgreich und komplett abschliessen konnte. Zum Schluss hatte ich noch etwas Probleme mit dem Latex-Build. Dieser hatte die falschen Commands ausgeführt. Ich konnte dies dann allerdings beheben.
	\\
	\hline
	\textbf{Tagesreflexion}              & Heute war nebst der Reflexion noch der Feinschliff anstehend. Am morgen begann ich mit der Reflexion, als diese geschrieben war, machte ich noch einen letzten Feinschliff. Deshalb ist im Zeitplan der Meilenstein E, etwas später fertig geworden als geplant. Ganz am Schluss musste ich noch den Anhang erstellen, und beides zusammen abgeben.
	\\
	\hline
	\textbf{In Anspruch genommene Hilfe} & keine \\ 
	\hline
\end{longtable}\label{tab:arbeitsprotokoll-tag10}
\newpage


