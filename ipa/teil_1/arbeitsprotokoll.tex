\chapter{Arbeitsprotokoll}\label{ch:arbeitsprotokoll}
\renewcommand{\arraystretch}{1.5}
\begin{longtable}{p{.22\textwidth}|p{.78\textwidth}}
    \hline
    \textbf{Datum}                       & 06.11.2024\\
    \hline
    \textbf{Bearbeitete Arbeitspakete}   & 1.1, 1.2, 2.1, 2.2, 7.1, 7.2, 7.3\\
    \hline
    \textbf{Arbeitszeit}                 & 8h \\
    \hline
    \textbf{Überzeit}                    & 0 \\
    \hline
    \textbf{Vergleich mit dem Zeitplan}  & Da ich den Zeitplan noch nicht fertig erstellt habe, kann ich für heute keinen Vergleich ziehen. \\
    \hline
    \textbf{Erfolge und Probleme}        & Zu Beginn wusste ich nicht genau wie ich am besten vorgehe resp. was ich zuerst angehe, da hat mir das vorhandene Template einen sehr guten Leitfaden gegeben. Und so habe ich begonnen alles der Reihe nach auszufüllen/ zu dokumentieren. Und bin am Schluss weiter gekommen als gedacht.\\
    \hline
    \textbf{Tagesreflexion}              & Heute bin ich sehr gut voran gekommen. Ich konnte bereits den Teil 1 der Dokumentation abschliessen und mit den Arbeitspaketen beginnen.
    \\
    \hline
    \textbf{In Anspruch genommene Hilfe} & Fragen an Pascal bezüglich der Aufgabenstellung. War mir unsicher, wo genau die Kundendoku hin muss. Jetzt weiss ich, dass es reicht, wenn ich sie im Anhang anhänge.\\
    \hline
\end{longtable}\label{tab:arbeitsprotokoll-...} 

\newpage

\begin{longtable}{p{.22\textwidth}|p{.78\textwidth}}
	\hline
	\textbf{Datum}                       & ... \\
	\hline
	\textbf{Bearbeitete Arbeitspakete}   & ... \\
	\hline
	\textbf{Arbeitszeit}                 & ... \\
	\hline
	\textbf{Überzeit}                    & ... \\
	\hline
	\textbf{Vergleich mit dem Zeitplan}  & ... \\
	\hline
	\textbf{Erfolge und Probleme}        & ...
	\\
	\hline
	\textbf{Tagesreflexion}              & ...\\
	\hline
	\textbf{In Anspruch genommene Hilfe} & ... \\
	\hline
\end{longtable}\label{tab:arbeitsprotokoll-...}

\newpage

\begin{longtable}{p{.22\textwidth}|p{.78\textwidth}}
	\hline
	\textbf{Datum}                       & ... \\
	\hline
	\textbf{Bearbeitete Arbeitspakete}   & ... \\
	\hline
	\textbf{Arbeitszeit}                 & ... \\
	\hline
	\textbf{Überzeit}                    & ... \\
	\hline
	\textbf{Vergleich mit dem Zeitplan}  & ... \\
	\hline
	\textbf{Erfolge und Probleme}        & ...
	\\
	\hline
	\textbf{Tagesreflexion}              & ...
	\\
	\hline
	\textbf{In Anspruch genommene Hilfe} & ... \\
	\hline
\end{longtable}\label{tab:arbeitsprotokoll-...}
\newpage
