
\chapter{Projektmanagementmethode}\label{ch:projektmanagementmethode}

In diesem Kapitel wird die Projektmanagementmethode IPERKA\cite{iperka} beschrieben. Es wird dargelegt wieso diese Methode gewählt wurde und was die Vor und/oder Nachteile daran sind.

\section{IPERKA}\label{sec:iperka}

Für die Probe-VA wurde IPERKA als Projektmanagementmethode gewählt. Sie eignet sich gut für kleine Projekte. Sie lassen sich damit einfach und strukturiert planen sowie umsetzen. 
Die IPERKA Methode setzt sich aus folgenden 6 Schritten zusammen:

\subsection{Informieren}
Der erste Punkt bei IPERKA ist das Informieren. Dabei wird sich ein Überblick über das Projekt / den Projektauftrag verschaffen. Es gilt zu klären was genau der Auftrag ist, und ob alle Informationen vorhanden sind.

\subsection{Planen}
Als zweiten Schritt kommt das Planen. Hier wird das Projekt konkreter und es wird ein Zeitplan erstellt. Es werden Lösungsvorschläge erarbeitet und je nach Team grösse, werden bestimmte Aufgaben zugeteilt. Im Probe-IPA Fall fällt die Aufgabenverteilung natürlich weg. 

\subsection{Entscheiden}
Beim Entscheiden, wird entschieden welchen Lösungsweg gegangen werden soll. Es werden die verschiedenen Lösungsvarianten mit einander verglichen und evaluiert, welche die beste ist. Des weiteren wird z.B definiert mit welchen Tools / Technologien gearbeitet wird. Im Fall dieser Probe-IPA ist dies durch das bereits bestehende Projekt schon vorgegeben. Wichtig ist auch, dass die Kriterien, welche zu diesen Entscheidungen geführt haben, definiert werden.

\subsection{Realisieren}
In diesem Teil geht es an die Umsetzung. Das Projekt wird nach dem definierten Plan sowie Zeitplan versucht umzusetzen.

\subsection{Kontrollieren}
Der fünfte Schritt erfolgt teilweise parallel zum Vierten. In diesem Schritt wird von oben auf das laufende Projekt geblickt und geschaut, ob alles nach Plan läuft. Gibt es Abweichungen und falls ja, können diese begründet werden?

\subsection{Auswerten}
Der letzte Schritt dient dazu, nochmals auf das Projekt zurückzublicken und es zu Reflektieren.

\section{Alternative Methode - Scrum}\label{sec:alternative-methode}
Nebst IPERKA gibt es auch alternative Projektmanagementmethoden. Eine davon ist Scrum \cite{scrum}. 
Scrum eignet sich allerdings nicht besonders für die Umsetzung eines Projekts wie die Probe-IPA.  Sie ist eine Agile Projektmanagementmethode, welche sich für Projekte eignet, die sehr dynamisch und doch komplex sind. Meistens sind die konkreten Anforderungen zu Beginn sogar noch unklar. Zudem kann Scrum nur teilweise alleine durchgeführt werden.
Dies ist bei der Probe-IPA nicht der Fall. Deshalb wurde sich für IPERKA entschieden.